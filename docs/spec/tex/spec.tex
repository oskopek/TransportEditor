\documentclass[10pt,a4paper,oneside]{article}

% This first part of the file is called the PREAMBLE. It includes
% customizations and command definitions. The preamble is everything
% between \documentclass and \begin{document}.

\usepackage[margin=1in]{geometry}  % set the margins to 1in on all sides
\usepackage{graphicx}              % to include figures
\usepackage{amsmath}               % great math stuff
\usepackage{amsfonts}              % for blackboard bold, etc
\usepackage{amsthm}                % better theorem environments
\usepackage[english]{babel}
\usepackage[square, numbers]{natbib}
\usepackage[T1]{fontenc}
\usepackage[utf8]{inputenc}
\usepackage{lmodern}
\usepackage{amssymb}
\usepackage{authoraftertitle}
\usepackage{hyperref}
\usepackage{multicol}
\usepackage{caption}
\usepackage{subcaption}
\usepackage{placeins}
\usepackage{setspace}

%\onehalfspace
\singlespace

% Author
\author{Ondrej Škopek\\
Faculty of Mathematics and Physics\\
Charles University in Prague\\
\texttt{\href{mailto:oskopek@matfyz.cz}{oskopek@matfyz.cz}}}

\title{\textbf{DRAFT:} Individual software project -- specification -- CodeName:TEdit}

%\date{3. mája 2015}
\date{\today}

% hyperref

\hypersetup{
    bookmarks=true,         % show bookmarks bar?
    unicode=true,          % non-Latin characters in Acrobat’s bookmarks
    pdftoolbar=true,        % show Acrobat’s toolbar?
    pdfmenubar=true,        % show Acrobat’s menu?
    pdffitwindow=true,     % window fit to page when opened
    pdfstartview={FitV},    % fits the width of the page to the window
    pdftitle={\MyTitle},    % title
    pdfauthor={\MyAuthor},     % author
    pdfsubject={\MyTitle},   % subject of the document
    pdfcreator={\MyAuthor},   % creator of the document
    pdfproducer={\MyAuthor}, % producer of the document
    pdfkeywords={software} {project} {planning} {automated planning} {java} {javafx} {ipc} {transport domain}, % list of keywords
    pdfnewwindow=true,      % links in new PDF window
    colorlinks=false,       % false: boxed links; true: colored links
    linkcolor=red,          % color of internal links (change box color with linkbordercolor)
    citecolor=green,        % color of links to bibliography
    filecolor=magenta,      % color of file links
    urlcolor=cyan,           % color of external links
}


% various theorems, numbered by section

\newtheorem{thm}{Theorem}[section]
\newtheorem{lem}[thm]{Lemma}
\newtheorem{obsv}[thm]{Observation}
\newtheorem{cor}[thm]{Corollary}
\newtheorem{conj}[thm]{Conjecture}

\DeclareMathOperator{\id}{id}

\newcommand{\TODO}[1]{{\textbf{TODO:} #1}} % for TODOs
\newcommand{\comment}[1]{} % for comments

\newcommand{\dist}{\text{dist}} % distance function
\newcommand{\bd}[1]{\mathbf{#1}} % for bolding symbols 
\newcommand{\RR}{\mathbb{R}}      % for Real numbers
\newcommand{\ZZ}{\mathbb{Z}}      % for Integers
\newcommand{\col}[1]{\left[\begin{matrix} #1 \end{matrix} \right]}
\newcommand{\comb}[2]{\binom{#1^2 + #2^2}{#1+#2}}
\newcommand{\pname}{Codename:TEdit } % project name (or code name)

\begin{document}
\maketitle

\TODO Vymysliet nazov programu: TransportEditor, TEdit, VisualTransport











\section{Basic information}

\pname aims to be a problem editor and plan visualizer for the Transport domain from the International Planning Competition 2008.
The goal is to create an intuitive GUI desktop application for making quick changes and re-planning, but also designing a new problem dataset from scratch. \pname will help researchers working on this domain fine-tune their planners; they can visualize the various corner cases their planner fails to handle, step through the generated plan and find the points where their approach fails.
A secondary motivation is to be able to test approaches for creating plans for the domain as part of our future bachelor thesis.

\subsection{The Transport planning domain}

Transport is a domain designed originally for the International Planning Competition (IPC, part of the International Conference on Automated Planning and Scheduling ICAPS).
Originally, Transport appeared at \href{http://icaps-conference.org/ipc2008/deterministic/Domains.html}{IPC-6 2008}.
Since then, it has been used in every IPC, specifically \href{http://www.plg.inf.uc3m.es/ipc2011-deterministic/}{IPC-7 2011}
and \href{https://helios.hud.ac.uk/scommv/IPC-14/}{IPC-8 2014}.

There are two basic formulations of the Transport domain family (i.e. two ``similar Transport domains''):
\begin{itemize}
\item \verb+transport-strips+ -- the classical, sequential Transport domain. See section \ref{transport-strips}.
\item \verb+transport-numeric+ -- the numerical Transport domain. See section \ref{transport-numeric}.
\end{itemize}

Both of these formulations have been used interchangeably in various competition tracks.
The following is an overview of the distinct datasets, their associated IPC competition, track at the competition and the formulation used (descriptions of the tracks in hyperlinks):

\begin{center}
\begin{tabular}{|c|c|c|c|}
\hline 
Dataset name & Competition & Track & Formulation \\ 
\hline 
netben-opt-6 & IPC-6 & \href{http://icaps-conference.org/ipc2008/deterministic/NetBenefitOptimization.html}{Net-benefit: optimal} & Numeric \\ 
\hline 
seq-opt-6 & IPC-6 & \href{http://icaps-conference.org/ipc2008/deterministic/SequentialSatisficing.html}{Sequential: satisficing} & STRIPS \\ 
\hline 
seq-sat-6 & IPC-6 & \href{http://icaps-conference.org/ipc2008/deterministic/SequentialOptimization.html}{Sequential: optimal} & STRIPS \\ 
\hline 
tempo-sat-6 & IPC-6 & \href{http://icaps-conference.org/ipc2008/deterministic/TemporalSatisficing.html}{Temporal: satisficing} & Numeric \\ 
\hline 
seq-agl-8 & IPC-8 & \href{https://helios.hud.ac.uk/scommv/IPC-14/seqagi.html}{Sequential: agile} & STRIPS \\ 
\hline 
seq-mco-8 & IPC-8 & \href{https://helios.hud.ac.uk/scommv/IPC-14/seqmulti.html}{Sequential: multi-core} & STRIPS \\ 
\hline 
seq-opt-8 & IPC-8 & \href{https://helios.hud.ac.uk/scommv/IPC-14/seqopt.html}{Sequential: optimal} & STRIPS \\ 
\hline 
seq-sat-8 & IPC-8 & \href{https://helios.hud.ac.uk/scommv/IPC-14/seqsat.html}{Sequential: satisficing} & STRIPS \\ 
\hline 
\end{tabular} 
\end{center}

Short descriptions of the various tracks and subtracks can be found in the rule pages of
\href{https://helios.hud.ac.uk/scommv/IPC-14/rules.html}{IPC-6}
and the  \href{http://icaps-conference.org/ipc2008/deterministic/CompetitionRules.html}{rule page of IPC-8}.

Unfortunately, we weren't able to acquire the datasets for IPC-7, as the Subversion repository that promises to contain them is unavailable (\url{http://www.plg.inf.uc3m.es/ipc2011-deterministic/Domains.html}).

% Comparison with Trucks from IPC-5
%In general, the Transport domain family is very similar to the Trucks domain family from \href{http://idm-lab.org/wiki/icaps/ipc2006/deterministic/}{IPC-5 2006}. The key differences are ...

\subsection{Transport STRIPS formulation description}\label{transport-strips}

The STRIPS version of Transport is a logistics domain -- vehicles with limited capacities drive around on a (generally asymmetric) positively-weighted oriented graph, picking up and dropping packages along the way. Picking up or dropping a package costs 1, driving along a road costs depending on the edge weight. All packages have a size of 1. The general aim is to minimize the total cost, while delivering all packages to their destination.

\subsection{Transport Numeric formulation description}\label{transport-numeric}

The numerical version of Transport is very similar to the STRIPS version, see section \ref{transport-strips}. The key differences are:
\begin{itemize}
\item Package sizes can now be any positive number.
\item The concept of fuel -- every vehicle has a maximum fuel level, current fuel level, and all roads have a fuel demand (generally different than the length of the road). A vehicle can refuel if there is a petrol station at the given location. Refuelling always fills the vehicle's tank to the max.
\item The introduction of time:
\begin{itemize}
\item The duration of driving along a road is equal to it's length.
\item The duration of picking a package up or dropping it off is equal to 1.
\item The duration of refuelling is equal to 10.
\item A vehicle cannot pick up or drop packages concurrently -- it always handles packages one at a time.
\item A vehicle cannot do other actions during driving to another location (it is essentially placed ``off the graph'' for the duration of driving).
\end{itemize}
\item The cost function is removed (we now minimize the total duration of a plan).
\end{itemize}


















\section{Feature requirements}

In this section, we present the basic functionality requirements for \pname.

\subsection{Functionality overview}

The basic work flow of \pname consists of the following user's steps:
\begin{itemize}
\item Select which formulation of the Transport domain they want to work with.
\item Load a problem of the given domain. See section \ref{inputoutput} for details on the input format.
\item \pname draws the given graph as good as it can.
\item Iterate among the following options:
\begin{itemize}
\item Load a planner executable and let \pname run the planner on the loaded problem instance for a given time, then load the resulting plan.
\item Load a pre-generated plan.
\item Step through the individual plan actions and let \pname visualize them.
\item Edit the graph: add/remove/edit the location or properties of vehicles, packages, roads, locations and possibly petrol stations.
\item Save the currently generated plan.
\item Save the problem (along with the graph drawing hints). For details, see section \ref{inputoutput}.
\end{itemize}
\item Save and close the currently loaded problem. Exit the application, or go back to the first step.
\end{itemize}

There are a lot of requirements that arise from the typical work flow above. We will describe them in the next few sections.

\subsection{User interface functionality}

In order for \pname to be useful to it's users, it has to have enable a quick and efficient work flow.
The biggest challenge will be to design an intuitive and responsive graphical user interface.
Key parts of the user interface are:
\begin{itemize}
\item A large and not crowded drawing of the road graph, showing only the most relevant information, possibly letting the user display details on-demand.
\item A clear and concise list of actions in the plan the user wants to visualize.
\item A panel of tools to edit the problem and graph with.
\end{itemize}

We will describe these three in detail in the following sections.

\subsubsection{Graph visualization} \label{graphviz}

Above all, the graph should be reasonably well arranged and drawn. However, drawing an arbitrary graph is considered a hard problem on it's own. \TODO citation needed
We will focus on implementing a reasonable graph drawing algorithm and let the user make manual drag-and-drop style changes on top of the graph drawing we produce.

A more specific property of the Transport domain graphs is that they have quite a lot of data on the edges (roads) and vertexes (locations) -- road lengths, fuel demands, location names and petrol station, package and vehicle current locations.
Clearly, this cannot fit on the graph drawing at once -- we will have to do selective drawing of the information in order
not to crowd the graph.
Popping up overlays on user mouse-over
or a graph legend where the user can select what info to currently show are just a few options.

\subsubsection{Plan action list}

An important part of the visualization work flow is the actual planner-generated plan.
The user interface will show a context-aware list of them.

Because plans can be quite long (generally hundreds/thousands of actions),
we cannot show them all at once.
The list will only show the currently selected action, a few actions preceding it and a few upcoming actions.

The graph described in section \ref{graphviz} and the state it represents will change dynamically with the selected action in the list. The user can fluently scroll through the plan actions, continually stepping through the plan either forwards or backwards.
Jumping to a specific action will also be supported.

\subsubsection{Editing tools}

A big feature of \pname is editing the domain problems on the fly.
All the graph vertexes will be movable using drag and drop (the edges will move along).
Details of individual elements on the graph (vehicles, packages, locations, roads) will be editable
on user request.

Additionally, a toolbox will be present. It will enable quick access to buttons for adding/removing locations, roads, vehicles and packages.

To prevent accidental changes to the problem when we just want to test plan generation,
the toolbox will contain a lock feature -- once enabled, no changes to the graph will be allowed.

\subsection{Performance}

The performance requirements of \pname cannot be numerically quantified -- the main benchmark is fast-enough
responsiveness to user actions. The implementation specific details on how to achieve this are discussed in the section \ref{used-tech}.

\subsection{Data model re-usability}

The internal data representation has to be reusable and therefore, reasonably decoupled from the rest of the application.
It is very probable that we will want to implement a planner for the Transport domain later on
-- and for that, we can reuse the entire back-end data model that is used to represent the domain problem in memory
and build the planner on top of it.
More details on the specifics of decoupling of individual modules of \pname are in section \ref{modules}.

\subsection{Optional features}

A few more optional features, which may or may not be included in the final product:

\begin{itemize}
\item Automatic plan verification -- Upon loading a plan, verify (using VAL) that the plan is valid for the given problem and domain.
\item \TODO add more optional features
\end{itemize}









\section{Program decomposition}

In this section, we present a basic decomposition of the application from the logical perspective.

\subsection{Modules} \label{modules}

The application will be split up into individual modules.
Modules should be decoupled as much as possible, communicating only through well specified interfaces.
We propose the following modules:

\begin{itemize}
\item View -- representation of the individual UI elements.
\item Model-View -- an interfacing module, sitting between the view and model. Serves as a mediator -- updating the view based on model changes, and propagating user actions from the view back into the model.
\item Model -- the back-end data model, an internal problem data representation.
\item Persistence -- handles saving/loading the model and plans for the model to an external data store (f.e.~hard disk).
\end{itemize}

\subsection{UML diagram of the model}

\TODO UML diagram
 
 
 
 
 



\section{Software description}

In this section, we describe the chosen technology and how we aim to fulfil the requirements using it.

\subsection{Used technologies} \label{used-tech}

We have to chosen to implement the application along with the graphical user interface as a
desktop Java 8 program. The main reasons for this choice were portability and our experiences working in Java and it's associated technologies.

The reason for choosing Java 8 as the minimum version is JavaFX 8 -- a successor of JavaFX 2.0, now newly incorporated into
the standard Java Runtime Environment (JRE) and available on all Java Standard Edition (Java SE) versions of the JRE.

The model will use features not specific to JavaFX and therefore be reusable in any standard Java SE 8 and newer program.

We will probably use multiple open-source libraries which will all be specified in the developer documentation (section \ref{docs})
to write cleaner, more expressive code.

The application will be shipped as a ZIP archive, containing an executable JAR file and the user documentation.

\subsubsection{Environment dependencies}

The runtime dependencies that are implied by the previous section are:
\begin{itemize}
\item Java SE, version 8+
\end{itemize}
All other dependencies will be packaged and shipped along with the product.

The development dependencies will be:
\begin{itemize}
\item Java SE, version 8+
\item Maven, version 3+
\end{itemize}

Any hardware dependencies or arising software dependencies will be specified in the user or developer documentation (section \ref{docs}).

\subsection{User and developer documentation} \label{docs}

We hope most of the user documentation will be left unread -- the user interface and work flow should be
mostly self-explainable and intuitive.
Nevertheless, we will produce user documentation, involving:
\begin{itemize}
\item a help document accessible from the menu bar inside the application
\item helpful dialogs or other pop-up messages
\item a standalone text document, documenting the features, supported data formats, instructions on the installing Java JRE and a demo usage tutorial
\end{itemize}

The developer documentation consists of in-code comments, JavaDoc comments (and a generated HTML page hierarchy using the \verb+javadoc+ tool) and a standalone text file, containing information on:
\begin{itemize}
\item Links to the code repository, issue tracker, continuous integration service, \ldots
\item Getting the code from the code repository.
\item Installing the needed dependencies.
\item Compiling and building the project.
\item Finding their way around the project: a high-level overview of the project, module descriptions, \ldots
\end{itemize}






\section{Input \& output format} \label{inputoutput}


\TODO an extension to the input PDDL in comments ;coordinates










\section{User interface}


\subsection{Main screen}

\FloatBarrier
\begin{figure}
        \centering
        \includegraphics[width=0.8\textwidth]{../data/img/pdf/gui}
        \caption{Abstract GUI prototype}
        \label{fig:gui}
\end{figure}
\FloatBarrier

\TODO Need to do selective label drawing (too crowded) with popup info boxes
\TODO Make the graph nodes moveable
\TODO scrollable own widget (transparent, fluid) for the plan. --> partial order plans? No.
\TODO How do we simulate movement on an abstract graph?

\subsection{Helper screens}















\section{Open questions and ending notes}

\begin{itemize}
    \item Do we support the numerical Transport domain too?
    \item Is it safe to assume that symmetric edges are weight-symmetric? (all datasets have this) -- more of a planner question
\end{itemize}

The conversion from hand-drawn pictures to vector images was done using the wonderful
open-source project cartoonist: \url{https://github.com/honzajavorek/cartoonist}.

{
\footnotesize % 10pt in 12pt article size
\begingroup
\bibliographystyle{plainnat}
\bibliography{bibtex}
\endgroup

}
\end{document}
\begin{tabular}{cccc}
\toprule
\textbf{\#} & \textbf{MSFA3} & \textbf{MSFA5} & \textbf{RRAPN}\\
\midrule
p01 & 0.82 & 0.82 & 0.82\\
p02 & 1.00 & 1.00 & 0.92\\
p03 & 0.82 & 0.82 & 0.86\\
p04 & 0.99 & 1.00 & 0.68\\
p05 & 1.00 & 0.94 & 0.72\\
p06 & 0.99 & 1.00 & 0.68\\
p07 & 1.00 & 1.00 & 0.84\\
p08 & 0.89 & 0.90 & 0.85\\
p09 & 1.00 & 1.00 & 0.84\\
p10 & 1.00 & 1.00 & 0.78\\
p11 & 0.96 & 0.96 & 0.86\\
p12 & 1.00 & 1.00 & 0.76\\
p13 & 0.68 & 0.68 & 0.75\\
p14 & 0.83 & 0.83 & 0.88\\
p15 & 0.69 & 0.69 & 0.67\\
p16 & 0.91 & 0.92 & 0.68\\
p17 & 0.94 & 0.98 & 0.85\\
p18 & 1.00 & 1.00 & 0.81\\
p19 & 1.00 & 1.00 & 0.79\\
p20 & 0.92 & 0.98 & 0.77\\
\bottomrule
\end{tabular}


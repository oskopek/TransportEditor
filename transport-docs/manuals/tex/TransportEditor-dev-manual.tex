\documentclass[12pt,a4paper,twoside]{article}
\setlength\textwidth{160mm}
\setlength\textheight{247mm}
\setlength\oddsidemargin{0mm}
\setlength\evensidemargin{0mm}
\setlength\topmargin{0mm}
\setlength\headsep{0mm}
\setlength\headheight{0mm}

%% Generate PDF/A-2u
\usepackage[a-2u]{pdfx}

%% Character encoding: usually latin2, cp1250 or utf8:
\usepackage[utf8]{inputenc}

\hyphenation{Transport-Editor}

%% Prefer Latin Modern fonts
\usepackage{lmodern}

%% Further useful packages (included in most LaTeX distributions)
\usepackage{amsmath}        % extensions for typesetting of math
\usepackage{amsfonts}       % math fonts
\usepackage{amsthm}         % theorems, definitions, etc.
\usepackage{bbding}         % various symbols (squares, asterisks, scissors, ...)
\usepackage{bm}             % boldface symbols (\bm)
\usepackage{graphicx}       % embedding of pictures
\usepackage{fancyvrb}       % improved verbatim environment
\usepackage{natbib}         % citation style AUTHOR (YEAR), or AUTHOR [NUMBER]
\setcitestyle{round} % TODO: round brackets for citep and citet
\usepackage[nottoc]{tocbibind} % makes sure that bibliography and the lists
			    % of figures/tables are included in the table
			    % of contents
\usepackage{dcolumn}        % improved alignment of table columns
\usepackage{booktabs}       % improved horizontal lines in tables
\usepackage{paralist}       % improved enumerate and itemize
\usepackage[usenames]{xcolor}  % typesetting in color

\title{TransportEditor Developer Manual}
\author{Ondrej~{\v{S}}kopek}
\date{\today}

%% The hyperref package for clickable links in PDF and also for storing
%% metadata to PDF (including the table of contents).
%% Most settings are pre-set by the pdfx package.
\hypersetup{unicode}
\hypersetup{breaklinks=true}

% Title page and various mandatory informational pages
\begin{document}
\maketitle

%%%%%%%%% README
TransportEditor aims to be a problem editor and plan visualizer for the Transport domain from the IPC 2008.
The goal is to create an intuitive GUI desktop application for making quick changes and re-planning,
but also designing a new problem dataset from scratch.

\section{Getting help}

\textbf{For users looking for help}: a manual describing all the available features of TransportEditor is available in the app itself:
in the upper menu bar item \textbf{Help $\to$ Help}.

Post any development questions or comments you may have on Stack Overflow and/or don't hesitate to
open an issue at \url{https://gitlab.com/oskopek/TransportEditor/issues}.

\section{Running TransportEditor releases}
\begin{itemize}
\item Download a release from:\\ \url{https://github.com/oskopek/TransportEditor/releases}

\item Note: TransportEditor uses semantic versioning (\url{http://semver.org/}).

\item To directly run TransportEditor, download the \textbf{executable JAR file}:\\ \texttt{TransportEditor-VERSION-jar-with-dependencies.jar}

\item If you want to run a release, just try:\\ \texttt{java -jar TransportEditor-VERSION-jar-with-dependencies.jar}
\end{itemize}

\section{Building \& running TransportEditor}\label{readme-building}

\begin{itemize}
\item See the section (further down) on How-to setup your \textbf{build environment} first.

\item \textbf{Recommended}: \texttt{mvn clean install -DskipTests}

\item To run \textbf{unit tests}: \texttt{mvn clean install}

\item To run \textbf{integration tests}: \texttt{mvn clean install -Pit}

\item To \textbf{clean}, run: \texttt{mvn clean}

\item To build \textbf{docs} too: \texttt{mvn clean install -Pdocs}

\item \textbf{Run TransportEditor}:
\begin{itemize}
\item If you followed the build environment setup and you want to run your version of TransportEditor,
run the command \texttt{mvn exec:java} from the \texttt{transport-editor} module directory.

\item If you want to run a translated version of TransportEditor, set your system locale accordingly and restart TransportEditor. If you want to try out a translated version (on Linux), try:\\ \verb+LC_ALL="LOCALE" java -jar TransportEditor.jar+\\
Substitute \texttt{LOCALE} for any locale available on your system. To view all available locales,
run \verb+locale -a+.
\end{itemize}
\end{itemize}

\section{Setup your build environment}
\begin{enumerate}

\item \textbf{Install Git}

\begin{itemize}

\item Fedora: \texttt{sudo dnf install git}

\item Ubuntu: \texttt{sudo apt-get install git}

\item Windows: Download and install Git for Windows at:\\ \url{https://git-scm.com/downloads}

\end{itemize}

\item \textbf{Install git-lfs} from \url{https://git-lfs.github.com/}

\item \textbf{Install Java8 JDK} -- Oracle JDK Downloads\footnote{\url{http://www.oracle.com/technetwork/java/javase/downloads/index.html}} -- Select: Java Platform (JDK)

\begin{itemize}

\item \textbf{NOTE}: You need \texttt{jdk-8u40} or newer (JavaFX 8 dependency).

\end{itemize}

\item \textbf{Install Maven} -- preferably the latest version you can.
Usually, your distribution's package management repository is enough:

\begin{itemize}

\item Fedora: \texttt{sudo dnf install mvn}

\item Ubuntu: \texttt{sudo apt-get install maven}

\item Windows: See
Maven on Windows\footnote{\url{http://maven.apache.org/guides/getting-started/windows-prerequisites.html}} and 
Maven Downloads.\footnote{\url{http://maven.apache.org/download.cgi}}

\end{itemize}

\item \textbf{Fork the repository} -- Create a fork of the oskopek/TransportEditor repository\footnote{\url{https://gitlab.com/oskopek/TransportEditor/}} using the button under the project logo on GitLab, usually the fork will be called: \texttt{yourusername/TransportEditor}.

\item \textbf{Clone the your fork} -- run:\\ \texttt{git clone https://gitlab.com/yourusername/TransportEditor.git}\\
Or, preferably, use SSH:\\ \texttt{git clone git@gitlab.com:yourusername/TransportEditor.git}

\item \textbf{Pull the LFS files} -- run: \texttt{git lfs pull}

\item \textbf{Run the build} (see: \nameref{readme-building})
\end{enumerate}

\section{Short design description}
The model for the Transport domain is pretty complicated
because it handles:

\begin{itemize}
\item Multiple variants of the Transport domain

\item Planning and visualization with the same model
\end{itemize}

That's what this short section is for -- describing the ideas behind the model, so that reading the code
afterward is easier. The model is split into 4 parts:

\begin{itemize}
\item Session
\item Domain
\item Problem
\item Plan
\end{itemize}

\section{Plan}
The plan consists of an ordered list of actions.
There are two types of plans:

\begin{itemize}
\item Sequential -- these plans are strictly linear, actions do not overlap (imagine simple linked list).
\item Temporal -- every action in this plan has a time interval in which it takes place.
This plan is basically a set of intervals with associated actions. For storing it, we use an
Interval tree\footnote{\url{https://en.wikipedia.org/wiki/Interval_tree}},
which allows efficient access to actions given a time or time range.
\end{itemize}

\subsection{Visualizing plans}

There are currently two ways to visualize both plan types:

\begin{itemize}
\item Simple list -- both sequential and temporal versions look similar. Both are filterable by right clicking on the headers.
\begin{itemize}
\item Sequential: uses a simple drag-and-drop reorderable table of action arguments.
See the screenshot on the top of the README for a preview. Is redrawn completely after every change.

\item Temporal: in contrast to the sequential variant, this one cannot be reordered by dragging. The start times can, however,
be edited, which will result in the table reordering itself. Is not redrawn completely, adjusts its internal state and
redraws the necessary parts.
\end{itemize}

\item Gantt chart -- both sequential and temporal versions look alike, resembling an XY chart, the X-axis being the time
axis and the Y-axis having all action objects. Both are redrawn every time the plan changes or its filter
in the simple list is changed.
\begin{itemize}
\item Sequential: using it to visualize sequential plans is not very interesting, as it offers almost no added insights on top of the simple list

\item Temporal: when visualizing temporal plans with a Gantt, we can observe the synchronicity of planned actions
and, to some extent, the cooperation of individual actors
\end{itemize}
\end{itemize}

\subsection{Persisting plans}
Using string manipulation and built-in constants and format, it is persisted into a VAL-like format \citep[Figure~2]{Howey2003}.
For parsing, we assume a correct and valid VAL-like plan. A very simple string manipulation and Regex-based approach
is used for both temporal and sequential plans. Additionally, a simple ANTLR\footnote{\url{http://www.antlr.org/}} grammar
is used in some places. See the \texttt{persistence} package in \verb+transport-core+ for details.

\section{Problem}
The problem is basically a graph (with multiple possible ``layers'', f.e. fuel) and a vehicle and package map.

Currently, we use GraphStream\footnote{\url{http://graphstream-project.org/}} for both the data storage and visualization of the graph.
Apart from nodes and edge arrows, everything else is visualized as
``sprites\footnote{\url{http://graphstream-project.org/doc/Tutorials/Graph-Visualisation/\#sprites}}''.

Fuel is added as different graph edge type (\verb+FuelRoad+ instead of \verb+DefaultRoad+) and a domain label change
(see \verb+PddlLabel+s in the domain).
If the domain is fuel enabled, the fuel properties of locations, roads, and vehicles else will be displayed.

\subsection{Visualizing problems}

Problem visualization does not fundamentally differ between different domains and problems.
Some problem tooltips/properties might dis/appear when changing the domain type.

The graph is automatically laid out using a \texttt{SpringBox} algorithm from GraphStream
for a given time and then switched to a manual layout.

\subsection{Persisting problems}
Both rule pages of IPC-6\footnote{\url{http://icaps-conference.org/ipc2008/deterministic/CompetitionRules.html}}
and IPC-8\footnote{\url{https://helios.hud.ac.uk/scommv/IPC-14/rules.html}}
specify PDDL 3.1 as their official modeling language (language for domain
and problem specification).
Daniel L. Kovacs proposed an updated and corrected BNF (Backus-Naur Form)
definition of PDDL 3.1.\footnote{\url{https://helios.hud.ac.uk/scommv/IPC-14/repository/kovacs-pddl-3.1-2011.pdf}}

Using a Freemarker\footnote{\url{http://freemarker.org/}} template and a lot of string manipulation it is persisted into PDDL.
For parsing, we assume a correct and valid problem and use a formal grammar written in ANTLR to parse PDDL into a generated code structure provided by ANTLR and the \texttt{maven-antlr-plugin}. The same grammar as for
domains is used. See the \texttt{persistence} package of \verb+transport-core+ and the accompanying \texttt{src/main/antlr4} folder for details.

\section{Domain}
There is basically only one domain type: \texttt{VariableDomain} (we also have the notion of a \texttt{SequentialDomain},
but it is basically just an in-code hardcoded equivalent of loading the sequential Transport domain PDDL
into a \texttt{VariableDomain}).

The domain contains flags (labels), telling us which ``layers'' are enabled and which are not.
The UI, validator, IO, and planner all take these into account.
It also contains methods for action creation using their correct domain-specified definitions
and provides other useful data (predicates, functions, \ldots).

\subsection{Visualizing domains}

Domains are not visualizable per se.

\subsection{Persisting domains}
Using a Freemarker template and a lot of string manipulation it is persisted into PDDL.
For parsing, we assume a correct and valid problem and use a formal grammar written in ANTLR
to parse PDDL into a generated code structure provided by ANTLR and the \texttt{maven-antlr-plugin}. The same grammar as for
problems is used.

TransportEditor doesn't load the PDDL domain definitions directly -- those are already built-in.
We only read the domain files to check which subset of conditions the user has chosen to model.

In the UI, we can also create a domain using a popup dialog backed internally by a \texttt{VariableDomainBuilder}.
It is essentially switchboard for gathering the appropriate flags and other properties the domain should have.

\section{Session}
The session is where everything comes together. It keeps an instance of the domain, problem, and plan (and planner and
validator, \ldots). We can use it to reason about what actions can be executed in the UI with the currently loaded
objects and also as a quick persistence solution -- if you save a session, you can then load it next time and
do not have to open all the individual parts again.

\subsection{Visualizing sessions}

Sessions are visualized by visualizing all their (possible) parts.

\subsection{Persisting sessions}
Sessions are persisted automatically to XML using XStream\footnote{\url{https://x-stream.github.io/}}. This means, all its properties
should be reasonably serializable (by reflection).

\section{Planning}
Any class implementing the \texttt{Planner} interface can be set as the planner for a session and if it has all the properties
that are needed (domain \& problem), we can generate a plan using an instance of that class. TransportEditor supports
external (executable) planners out of the box, given that the executable adheres to a few rules (for details, see
\texttt{ExternalPlanner}). An end of planning event is raised after planning finishes, for UI redrawing purposes.

\section{Validation}
Any class implementing the \texttt{Validator} interface can be used as a validator for plans in a planning session.
Validation happens automatically after planning in a session or it can be triggered manually. There are different
validators with different strictness (used for different domain variants). Choosing a wrong combination of domain,
problem and validator might cause false positives or false negatives, make sure to read the documentation of the
individual validators. TransportEditor supports a popular external validator called VAL \citep{Howey2003} out of the box.

\section{General notes}
There are few other small features of the project worth mentioning.

\subsection{CDI \& the EventBus}
CDI (Context and Dependency Injection) using Weld\footnote{\url{http://weld.cdi-spec.org/}} is used for inversion of control
and for communication without tight coupling. Should only be used in the UI part of the project.

For event-driven communication on the front end, Guava's \texttt{EventBus} is used. Again, it enables persistent
reactive handling without tight coupling.

\subsection{JavaFX properties and bindings}
The JavaFX based UI makes heavy use of bindings and properties, essential features of JavaFX. They enable
reactive changes to the UI in an efficient manner but can be a bit tricky when reading code that uses them.
For even more power, we use the ControlsFX\footnote{\url{http://fxexperience.com/controlsfx/}} library, but try to avoid it,
if possible.

\subsection{Model immutability}
The model (mainly the \texttt{model} package in the module \verb+transport-core+) is designed to be immutable
(excluding a few exceptions). This makes it easier to reason about complex, possibly multithreaded operations
on top of it. This note is useful to keep in mind when reading code that changes the model data.

\subsection{Tests}
The project aims to be well tested and verified. To stick to these goals, we have several levels of tests,
that are run by a CI (Continuous Integration) system after every push and should also be run by developers
(at least) after every commit. The displayed test coverage in the README is calculated as an average of unit
and integration test coverage.

TransportEditor currently has 3 types of tests:

\begin{itemize}
\item Unit tests (\texttt{*Test.java}) -- simple and quick to run tests that test one thing and test it well.

\item Integration tests (\texttt{*IT.java}) -- complex tests that handle multiple moving parts at once. Usually involving IO or
other not easily mockable things. Try to avoid abusively writing these in favor of unit tests, if possible.

\item User interface tests (\texttt{*UI.java}) -- test the UI using TestFX\footnote{\url{https://github.com/TestFX/TestFX}}.
At the moment, they are under-represented and not run very often. Currently, CI doesn't run them by default.
\end{itemize}

\subsection{Comments, code style}
TransportEditor employs a rigorous code style checker called \texttt{checkstyle} that is run automatically at every build.
Please adhere to that style when extending/editing the code base. Multiple other unwritten and unspecified rules might
apply. Please, do not take any style comments personally -- they are in place so that the code remains intact and
readable in the long term.

As part of the \texttt{checkstyle} process, JavaDoc comments are enforced on every method and class (excluding tests).
They should briefly describe the design/implementation choices, \textbf{why} they were made and any useful examples and or
other quirks.


%%% Bibliography
\begingroup
\bibliographystyle{plainnat}
\bibliography{bibtex}
\endgroup


\end{document}

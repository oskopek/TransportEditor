%%%%%%%%%%%%%%%%%%%%%%%%%%%%%%%%%%%%%%%%%
% baposter Landscape Poster
% LaTeX Template
% Version 1.0 (11/06/13)
%
% baposter Class Created by:
% Brian Amberg (baposter@brian-amberg.de)
%
% This template has been downloaded from:
% http://www.LaTeXTemplates.com
%
% License:
% CC BY-NC-SA 3.0 (http://creativecommons.org/licenses/by-nc-sa/3.0/)
%
%%%%%%%%%%%%%%%%%%%%%%%%%%%%%%%%%%%%%%%%%

%----------------------------------------------------------------------------------------
%	PACKAGES AND OTHER DOCUMENT CONFIGURATIONS
%----------------------------------------------------------------------------------------

\documentclass[a0paper,fontscale=0.285]{baposter} % Adjust the font scale/size here

\usepackage{natbib}         % citation style AUTHOR (YEAR), or AUTHOR [NUMBER]
\setcitestyle{round} % round brackets for citep and citet

\usepackage{graphicx} % Required for including images
\graphicspath{{../img/}} % Directory in which figures are stored

\usepackage{amsmath} % For typesetting math
\usepackage{amssymb} % Adds new symbols to be used in math mode

\usepackage{booktabs} % Top and bottom rules for tables
\usepackage{enumitem} % Used to reduce itemize/enumerate spacing
\usepackage{palatino} % Use the Palatino font
\usepackage[font=small,labelfont=bf]{caption} % Required for specifying captions to tables and figures

\usepackage{multicol} % Required for multiple columns
\setlength{\columnsep}{1.5em} % Slightly increase the space between columns
\setlength{\columnseprule}{0mm} % No horizontal rule between columns

\newcommand{\compresslist}{ % Define a command to reduce spacing within itemize/enumerate environments, this is used right after \begin{itemize} or \begin{enumerate}
\setlength{\itemsep}{1pt}
\setlength{\parskip}{0pt}
\setlength{\parsep}{0pt}
}

\definecolor{lightblue}{rgb}{0.145,0.6666,1} % Defines the color used for content box headers

\title{\huge Planning for Transportation Problems} % Poster title

\author{Ondrej \v{S}kopek} % Author(s)

\newcommand{\insertdepartment}{Department of Theoretical Computer Science and Mathematical Logic}
\newcommand{\institute}{Faculty of Mathematics and Physics, Charles University} % Institution(s)

\begin{document}

\begin{poster}
{
headerborder=closed, % Adds a border around the header of content boxes
colspacing=1em, % Column spacing
bgColorOne=white, % Background color for the gradient on the left side of the poster
bgColorTwo=white, % Background color for the gradient on the right side of the poster
borderColor=white, % Border color
headerColorOne=lightblue, % Background color for the header in the content boxes (left side)
headerColorTwo=lightblue, % Background color for the header in the content boxes (right side)
headerFontColor=white, % Text color for the header text in the content boxes
boxColorOne=white, % Background color of the content boxes
textborder=rectangle, % Format of the border around content boxes, can be: none, bars, coils, triangles, rectangle, rounded, roundedsmall, roundedright or faded
eyecatcher=false, % Set to false for ignoring the left logo in the title and move the title left
headerheight=0.1\textheight, % Height of the header
headershape=rectangle, % Specify the rounded corner in the content box headers, can be: rectangle, small-rounded, roundedright, roundedleft or rounded
headerfont=\Large\bf\textsc, % Large, bold and sans serif font in the headers of content boxes
%textfont={\setlength{\parindent}{1.5em}}, % Uncomment for paragraph indentation
linewidth=2pt % Width of the border lines around content boxes
}
%----------------------------------------------------------------------------------------
%	TITLE SECTION 
%----------------------------------------------------------------------------------------
%
{\includegraphics[height=8em]{logo-en.pdf}} % First university/lab logo on the left
{\vspace{0.2em}\huge\bf\textsc{Planning for Transportation Problems}\vspace{0.2em}%
} % Poster title
{\textsc{Ondrej {\v{S}}kopek\\\large\insertdepartment{}\\\institute{}}} % Author names and institution
{\includegraphics[width=21em]{logo-en.pdf}} % Second university/lab logo on the right

%----------------------------------------------------------------------------------------
%	OBJECTIVES
%----------------------------------------------------------------------------------------

\headerbox{Introduction}{name=introduction,column=0,row=0,span=2}{

\begin{multicols}{2}
Automated planning has historically been focused on \textit{domain-independent} planning --- planning without
the use of specific knowledge about the problem's domain.
As \citet{Nau2007} states, this is mostly due to the research field of planning wanting to
establish itself generally --- focusing on a set of domains would not be useful for that.
They also believe that this bias against domain-dependent planning is not as useful anymore.
We can now benefit from the attained theoretical results and an advancement in computing power,
resulting in a wider range of practical problems that can now be solved using planning techniques.

In this thesis, we will study variants of a specific planning domain introduced in the 2008 International Planning Competition (IPC) called \textit{Transport}. It serves as an abstract representation of a family of related transportation problems
and an important benchmark for planning.

The Transport domain, in its basic form, consists of a road network with items located at specified locations. The items are to be delivered to their destinations
using a fleet of vehicles. Our aim is to deliver all items
with the least total cost, or in the shortest amount of time,
where the cost and/or duration of individual actions is dependent on the domain variant.

A natural interpretation of the Transport domain is that it represents a set of trucks
delivering packages. However,
the exact same domain formulation could be used to
model a ride-sharing service like Uber, where cars drive around a city, picking up and dropping off people along the way, or
a means of modeling a rush-hour scenario in a public transportation system.

We aim to show that domain-dependent planning has an important
role to play in the future and there are many problems yet to be solved,
despite the loss of generality when compared to domain-independent planning.
To show this, we will design custom planners
and try to come up with domain-specific heuristics and other features
to aid our planners in solving Transport problems as well and as fast as
possible.
\end{multicols}
}

%----------------------------------------------------------------------------------------
%	RESULTS
%----------------------------------------------------------------------------------------

\headerbox{Results}{name=results,column=0,span=3,below=introduction}{

\begin{multicols}{3}
\textbf{IPC 2008}

In the updated results of the sequential satisficing track of IPC 2008 published after the competition,
the overall winner LAMA (a Fast Downward based planner)
was hands-down the best planner on the sequential Transport domain, winning
with a total quality of $28.93/30$, where all other planners had less than $20/30$.
Only 5 plans generated by LAMA had a worse total cost than the best known plans.

After adding our planners to the results,
the total quality of LAMA drops to $24.43/30$,
because several larger problems were solved better than the
best known solution from IPC 2008.
Our best planner on the IPC 2008 dataset, RRAPN, achieves a total quality of $27.77/30$,
which is a slight improvement over LAMA and other planners. The biggest gain of RRAPN is in being able to calculate
solutions of larger problems fast, which can be observed on
the results on problems 7--10 or 25--27,
which are the largest problems.
On the other hand, RRAPN fails to achieve optimal scores
on some smaller problems like problem number 2 or 12,
due to its explicit nature.

MSFA3 and MSFA5 are quite similar both in their construction and results on this dataset.
They generally obtain better results than RRAPN on smaller problems
(problems 2--3, 21--22),
but they can generate very good results even on larger problems,
like 14--20 or 28--29.
The reason why RRAPN occasionally obtains better plans than the
admissible heuristic of MSFA5 is that we weight it
with weights greater than or equal to 1,
therefore, making the heuristic inadmissible.
Based on total quality, MSFA5 marginally comes out on top as the better one of the two MSFA planners on this dataset.
All three of our planners beat all planners from the original competition based on total quality.

\textbf{IPC 2011}

The 2011 competition featured 20 sequential Transport problems,
with 4 planners (dae\_yahsp, LAMA 2008 and 2011, and roamer) achieving a total quality of more than $15/20$.
Interestingly, LAMA 2008 was able to produce better plans than its 2011 version in 12 out of 20 problems. The overall winner on Transport in 2011, roamer, achieved comparable scores on most problems to both versions of LAMA.

RRAPN consistenly achieves better scores than all domain-independent planners from the original competition in 17 out of the 20 problems (problem 4, 5, and 6). This can again be attributed to the size
of the problems.

Even though RRAPN is better than the original planners more often than both MSFA planners,
they come out on top based on total cost.
The differences in performance between MSFA planners
is almost indistinguishable, even on individual problem instances.
Even more interesting, the problems
where they perform well are complementary to the ones where RRAPN performs well,
as is visible on the results of problems 3--6, 10--12, and 13--15.

\textbf{IPC 2014}

In the sequential satisficing track of IPC 2014, the winner on the Transport domain
was without a doubt the Mercury planner, achieving
a stunning $20/20$ total quality. Even more interesting is the fact that
the runner-up yahsp3-mt achieved a score of only $10.74/20$
and all other planners achieved sub $10/20$ total quality,
accentuating the performance of Mercury even more.

After adding the results of our planners to the quality table,
the total quality of yahsp3-mt is lowered to $10.29/20$.
Mercury loses its spotless results, but still significantly dominates all
other planners, including ours, at $19.25/20$.

RRAPN manages to outperform yahsp3-mt with $15.80/20$, yet it fails
to match the results of Mercury, not even in one problem.
Both MSFA planners outperform RRAPN on this dataset with qualities around $18.50/20$,
but still do not come reasonably close to beating Mercury.
However, they do (marginally) outperform Mercury on some problems, like
problems 4--7, 9--10, 12, and 18--19.
The results of MSFA3 and MSFA5 on this dataset are almost identical.

\begin{center}
\includegraphics[width=\linewidth]{../../bp/imga/seq-sat-6-quality.pdf}
\captionof{figure}{seq-sat-6 quality}
\end{center}

\begin{center}
\includegraphics[width=\linewidth]{../../bp/imga/tempo-sat-6-quality.pdf}
\captionof{figure}{tempo-sat-6 quality}
\end{center}

\begin{center}
\includegraphics[width=\linewidth]{../../bp/imga/seq-sat-7-quality.pdf}
\captionof{figure}{seq-sat-7 quality}
\end{center}

\begin{center}
\includegraphics[width=\linewidth]{../../bp/imga/seq-sat-8-quality.pdf}
\captionof{figure}{seq-sat-8 quality}
\end{center}

\end{multicols}
}

%----------------------------------------------------------------------------------------
%	REFERENCES
%----------------------------------------------------------------------------------------

\headerbox{References}{name=references,column=0,above=bottom,span=2}{

\renewcommand{\section}[2]{\vskip 0.05em} % Get rid of the default "References" section title
\nocite{*} % Insert publications even if they are not cited in the poster
\small{ % Reduce the font size in this block
\bibliographystyle{plainnat}
\bibliography{../../bp/en/bibliography-poster.bib} % Use sample.bib as the bibliography file
}}

%----------------------------------------------------------------------------------------
%	FUTURE RESEARCH
%----------------------------------------------------------------------------------------

\headerbox{Future Research}{name=futureresearch,column=2,span=1,above=references,below=results}{ % This block is as tall as the references block

There remain more approaches to apply to
to domain-specific planning:
\begin{itemize}\compresslist
\item \textit{Hierarchical Task Networks} use \textit{tasks} (sequences of operators)
to carry out some goal. This approach is one of the most used in practice today (ad-hoc planners we designed are conceptually similar).

\item \textit{Pointer Networks and Reinforcement Learning} use special architectures of neural networks to solve TSP instances in \citet{Bello2016}.

\item \textit{Learning a domain-specific heuristic function} using neural networks \citep{Chen2011}. This
approach aims to help solve the problem of coming up with a good heuristic for a domain (a similar approach is also used in DeepStack \citep{Moravcik2017}).
\end{itemize}

}

%----------------------------------------------------------------------------------------
%	CONCLUSION
%----------------------------------------------------------------------------------------

\headerbox{Conclusion}{name=conclusion,column=0,span=2,above=references,below=results}{
Conclusion
}

%----------------------------------------------------------------------------------------
%	Acknowledgements
%----------------------------------------------------------------------------------------

\headerbox{Acknowledgements}{name=ack,column=2,span=1,above=bottom,below=futureresearch}{
Thank you to prof.\;RNDr.\;Roman\;Bart{\'{a}}k, Ph.D., my advisor.

Provided access to computing facilities of MetaCentrum is greatly appreciated.
}

%----------------------------------------------------------------------------------------
%	MATERIALS AND METHODS
%----------------------------------------------------------------------------------------

\headerbox{Materials \& Methods}{name=method,column=2,row=0}{ % This block's bottom aligns with the bottom of the conclusion block

Using several variants
of this domain, we will compare the performance of our custom-built planners to that of the planners taking part in the original competition and discuss various advantages or shortcomings of these approaches.
}


%----------------------------------------------------------------------------------------

\end{poster}

\end{document}

\begin{tabular}{|l|rrrrr|r|}
\hline
\textbf{Problem} & \textbf{RRAPNSched} & \textbf{TFD2014} & \textbf{base} & \textbf{sgplan6} & \textbf{tfd} & \textbf{BEST}\\
\hline
p01 & {\footnotesize 52} \textbf{1.00} & {\footnotesize 52.02} \textbf{1.00} & {\footnotesize 52} \textbf{1.00} & {\footnotesize 52} \textbf{1.00} & {\footnotesize 52} \textbf{1.00} & 52\\
p02 & {\footnotesize 136.01} \textbf{0.90} & uns. & {\footnotesize 217} \textbf{0.57} & {\footnotesize 217} \textbf{0.57} & {\footnotesize 208} \textbf{0.59} & 123\\
p03 & {\footnotesize 208.01} \textbf{0.91} & uns. & {\footnotesize 243} \textbf{0.78} & {\footnotesize 432} \textbf{0.44} & {\footnotesize 669} \textbf{0.28} & 189\\
p04 & {\footnotesize 270.02} \textbf{1.00} & {\footnotesize 425.29} \textbf{0.63} & uns. & {\footnotesize 845} \textbf{0.32} & uns. & 270.02\\
p05 & {\footnotesize 262.02} \textbf{1.00} & uns. & uns. & {\footnotesize 359} \textbf{0.73} & uns. & 262.02\\
p06 & {\footnotesize 257.02} \textbf{1.00} & {\footnotesize 408.31} \textbf{0.63} & uns. & {\footnotesize 965} \textbf{0.27} & uns. & 257.02\\
p07 & {\footnotesize 384.03} \textbf{1.00} & uns. & uns. & uns. & uns. & 384.03\\
p08 & {\footnotesize 377.04} \textbf{1.00} & uns. & uns. & uns. & uns. & 377.04\\
p09 & {\footnotesize 290.03} \textbf{1.00} & {\footnotesize 494.44} \textbf{0.59} & uns. & uns. & uns. & 290.03\\
p10 & uns. & uns. & uns. & uns. & uns. & --\\
p11 & uns. & uns. & {\footnotesize 629} \textbf{0.53} & {\footnotesize 629} \textbf{0.53} & {\footnotesize 549} \textbf{0.60} & 332\\
p12 & {\footnotesize 560.02} \textbf{0.77} & uns. & {\footnotesize 817} \textbf{0.53} & {\footnotesize 817} \textbf{0.53} & {\footnotesize 982} \textbf{0.44} & 433\\
p13 & uns. & {\footnotesize 1172.38} \textbf{0.33} & {\footnotesize 1216} \textbf{0.32} & {\footnotesize 650} \textbf{0.60} & {\footnotesize 3383} \textbf{0.11} & 389\\
p14 & {\footnotesize 691.03} \textbf{0.86} & {\footnotesize 1938.75} \textbf{0.31} & {\footnotesize 2059} \textbf{0.29} & uns. & uns. & 595\\
p15 & {\footnotesize 969.04} \textbf{0.85} & {\footnotesize 1143.45} \textbf{0.72} & uns. & {\footnotesize 2249} \textbf{0.37} & uns. & 824\\
p16 & {\footnotesize 840.04} \textbf{0.89} & {\footnotesize 2198.97} \textbf{0.34} & uns. & {\footnotesize 1875} \textbf{0.40} & uns. & 748\\
p17 & {\footnotesize 1038.05} \textbf{0.76} & {\footnotesize 2393.97} \textbf{0.33} & uns. & {\footnotesize 3331} \textbf{0.24} & uns. & 789\\
p18 & {\footnotesize 1333.07} \textbf{1.00} & uns. & uns. & uns. & uns. & 1333.07\\
p19 & {\footnotesize 1364.07} \textbf{1.00} & uns. & uns. & uns. & uns. & 1364.07\\
p20 & {\footnotesize 1664.09} \textbf{0.65} & uns. & uns. & {\footnotesize 6362} \textbf{0.17} & uns. & 1084\\
p21 & {\footnotesize 99.01} \textbf{0.64} & {\footnotesize 102.14} \textbf{0.62} & {\footnotesize 113} \textbf{0.56} & {\footnotesize 113} \textbf{0.56} & {\footnotesize 161} \textbf{0.39} & 63\\
p22 & {\footnotesize 134.02} \textbf{0.70} & {\footnotesize 265.38} \textbf{0.35} & {\footnotesize 238} \textbf{0.39} & {\footnotesize 238} \textbf{0.39} & uns. & 94\\
p23 & {\footnotesize 196.02} \textbf{0.63} & uns. & {\footnotesize 423} \textbf{0.29} & {\footnotesize 642} \textbf{0.19} & uns. & 123\\
p24 & {\footnotesize 214.02} \textbf{0.65} & uns. & {\footnotesize 1019} \textbf{0.14} & {\footnotesize 1116} \textbf{0.13} & uns. & 140\\
p25 & {\footnotesize 221.02} \textbf{0.71} & uns. & {\footnotesize 1404} \textbf{0.11} & uns. & uns. & 156\\
p26 & {\footnotesize 282.03} \textbf{1.00} & uns. & uns. & uns. & uns. & 282.03\\
p27 & {\footnotesize 312.03} \textbf{1.00} & uns. & uns. & uns. & uns. & 312.03\\
p28 & {\footnotesize 372.04} \textbf{1.00} & uns. & uns. & uns. & uns. & 372.04\\
p29 & {\footnotesize 382.04} \textbf{1.00} & uns. & uns. & uns. & uns. & 382.04\\
p30 & {\footnotesize 424.04} \textbf{1.00} & uns. & uns. & uns. & uns. & 424.04\\
\hline
\textbf{total} & \textbf{23.92} & \textbf{5.85} & \textbf{5.50} & \textbf{7.42} & \textbf{3.43} & \\
\hline
\end{tabular}


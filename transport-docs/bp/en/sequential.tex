\chapter{Sequential transportation planning}

\section{Specifics of sequential planning}

\TODO{mention that there are no vehicle goals.. verify!}

\section{Shortcommings of domain-independent approaches}

\TODO{Vytiahnut si plany z vysledkov seq-sat08 z LAMA na seqsat ipc08 an z mercury a lama na ipc14 + analysis of specific errors and shortcommings}

\section{Forward planning}

\TODO{in state-space, Making it deterministic -> Search}

\TODO{describe the alg}

\TODO{describe a generic search alg}

\subsection{Simple distance-based heuristics}

\TODO{revise}
When designing a heuristic, we want to provide an as precise as possible estimate
of the total plan cost or duration.
In \texttt{transport-strips}, the only thing we want is to deliver packages to their destinations. Therefore, a straightforward heuristic is to calculate the lengths of a shortest
path of each package to its destination using the road network and sum them.
To make it more precise, we will add 1 if the package is currently loaded
in a vehicle or 2 if it is not loaded and not yet at its destination.
This heuristic is definitely not optimal, meaning that there are situations,
where we will need actions summing up to a higher cost.
However, it is important to note, that the previously mentioned state space heuristic 
is not even admissible.

\TODO{Improve this heuristic by applying the shortest vehicle path too + extend it for vehicle goals}

An \textit{admissible} heuristic does not overestimate
the true value it is approximating. During planning in state space,
when examining a state $s$, we want to estimate the total costs of the best
plan getting us to a goal state from state $s$. In other words, because we are
trying to minimize the total cost,
a planning heuristic $h: S -> \N_0$ is admissible if and only if $\forall s \in S : h(s) \leq h^*(s),$
where $h^*$ is the true total cost. A similar definition is applicable for minimizing duration in the temporal variant.

Nonadmissible heuristics do not have nice properties when used with search algorithms
--- for example, they do not guarantee that the first path to a goal state we find
will be the optimal plan. To see how our previously constructed heuristic fails to do
this, imagine the situation of having a network with just two locations $A$ and $B$.
Two packages and one vehicle are located at $A$ and both packages want to be
transported to $B$. The road between $A$ and $B$ is symmetric and has length
of a 100. It is trivial to see that the optimal plan consists of two \pickup{} actions,
followed by a \drive{} and two \drop{}s. This plan has a total cost of $2+100+2=104$,
but the heuristic would estimate that we need 

\TODO{substitute drive actions for higher level drives to any node on graph, but forbid two sequential high level drives.. adv: always uses shortest paths, never cycles, minimal overhead.. disadv: might throw off cooperation w.r.t the heuristic?}

\section{Simple Greedy BFS}

\TODO{describe BFS, describe how it is complete but too slow, etc} \citep[Section~3.5]{Russell1995}

\TODO{mention Zhou with our Prolog \& Java impl} \citep{Zhou2015}

\section{SFA*}

\TODO{Describe A* in forward planning}

\TODO{Advantages and shortcommings of SFA*}



\subsection{SFA* with a package distance heuristic}

\subsection{SFA* with a different heuristic?}

\TODO{WA*, W=5 napr}

\subsection{SFA* with a different search (BFS/DFS, Beam)?}


\section{Backward planning?}

\subsection{Backward planning approach?}

\section{Some sort of a hierarchical planner?}

\section{CSP-based planner}\label{csp-approach}

\TODO{try to run a CSP in OptaPlanner to solve this and compare results, using Section~\ref{csp-custom-repr}}

\TODO{Advantages and shortcommings of a CSP-based planner}

\section{Summary and possible future approaches}

\TODO{Mention which planners we choose to compare in the experimental part}

\TODO{more possible approaches}

